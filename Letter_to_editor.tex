% Options for packages loaded elsewhere
\PassOptionsToPackage{unicode}{hyperref}
\PassOptionsToPackage{hyphens}{url}
%
\documentclass[
]{article}
\usepackage{lmodern}
\usepackage{amssymb,amsmath}
\usepackage{ifxetex,ifluatex}
\ifnum 0\ifxetex 1\fi\ifluatex 1\fi=0 % if pdftex
  \usepackage[T1]{fontenc}
  \usepackage[utf8]{inputenc}
  \usepackage{textcomp} % provide euro and other symbols
\else % if luatex or xetex
  \usepackage{unicode-math}
  \defaultfontfeatures{Scale=MatchLowercase}
  \defaultfontfeatures[\rmfamily]{Ligatures=TeX,Scale=1}
\fi
% Use upquote if available, for straight quotes in verbatim environments
\IfFileExists{upquote.sty}{\usepackage{upquote}}{}
\IfFileExists{microtype.sty}{% use microtype if available
  \usepackage[]{microtype}
  \UseMicrotypeSet[protrusion]{basicmath} % disable protrusion for tt fonts
}{}
\makeatletter
\@ifundefined{KOMAClassName}{% if non-KOMA class
  \IfFileExists{parskip.sty}{%
    \usepackage{parskip}
  }{% else
    \setlength{\parindent}{0pt}
    \setlength{\parskip}{6pt plus 2pt minus 1pt}}
}{% if KOMA class
  \KOMAoptions{parskip=half}}
\makeatother
\usepackage{xcolor}
\IfFileExists{xurl.sty}{\usepackage{xurl}}{} % add URL line breaks if available
\IfFileExists{bookmark.sty}{\usepackage{bookmark}}{\usepackage{hyperref}}
\hypersetup{
  pdftitle={Letter to editor},
  pdfauthor={Siddarth David; Chandrika Verma},
  hidelinks,
  pdfcreator={LaTeX via pandoc}}
\urlstyle{same} % disable monospaced font for URLs
\usepackage{graphicx,grffile}
\makeatletter
\def\maxwidth{\ifdim\Gin@nat@width>\linewidth\linewidth\else\Gin@nat@width\fi}
\def\maxheight{\ifdim\Gin@nat@height>\textheight\textheight\else\Gin@nat@height\fi}
\makeatother
% Scale images if necessary, so that they will not overflow the page
% margins by default, and it is still possible to overwrite the defaults
% using explicit options in \includegraphics[width, height, ...]{}
\setkeys{Gin}{width=\maxwidth,height=\maxheight,keepaspectratio}
% Set default figure placement to htbp
\makeatletter
\def\fps@figure{htbp}
\makeatother
\setlength{\emergencystretch}{3em} % prevent overfull lines
\providecommand{\tightlist}{%
  \setlength{\itemsep}{0pt}\setlength{\parskip}{0pt}}
\setcounter{secnumdepth}{-\maxdimen} % remove section numbering
\usepackage{booktabs}
\usepackage{longtable}
\usepackage{array}
\usepackage{multirow}
\usepackage{wrapfig}
\usepackage{float}
\usepackage{colortbl}
\usepackage{pdflscape}
\usepackage{tabu}
\usepackage{threeparttable}
\usepackage{threeparttablex}
\usepackage[normalem]{ulem}
\usepackage{makecell}
\usepackage{xcolor}

\title{Letter to editor}
\author{Siddarth David \and Chandrika Verma}
\date{}

\begin{document}
\maketitle

\hypertarget{background}{%
\section{Background}\label{background}}

A recent paper from Malawi compared the risk of mortality among
vehicular trauma patients transported by different modes of transport
{[}1{]}. After adjusting for injury severity, it reported higher risk of
mortality among patients transferred by police vehicles. Research from
other low-resource settings also show a higher mortality rate among
trauma patients conveyed by police transportation.

We used the TITCO-India data set, based on data from four tertiary care
hospitals across urban India (collected between 2013-2015), to measure
the risk of mortality by mode of transportation to the hospital in
directly admitted vehicular trauma patients. We estimated the relative
risk of mortality using a Poisson multivariate regression. We adjusted
for age, sex, and trauma severity using Injury Severity Score (ISS).

\begin{table}

\caption{\label{tab:unnamed-chunk-2}Table 1: Demographic Characterstics of directly admitted Vehicular Trauma in TITCO-India Data set}
\centering
\begin{tabular}[t]{l|l}
\hline
Variable & Summary\\
\hline
Gender (female \%) & 13.66\\
\hline
Age, years, mean (SD) & 35.4 (12.9)\\
\hline
ISS, mean (SD) & 11.6 (7.01)\\
\hline
\textbf{Mortality (\%)} & \textbf{15.21}\\
\hline
\end{tabular}
\end{table}

\begin{table}

\caption{\label{tab:unnamed-chunk-2}Table 2: Poisson multivariate regression for mortality adjusted for Age, sex, ISS}
\centering
\begin{tabular}[t]{l|l|l|l}
\hline
Mode.of.Transport & Relative.risk & CI..95... & p.value\\
\hline
Ambulance & Ref & y & x\\
\hline
Others & 0.000003 & y & x\\
\hline
Private vehicle & 0.78 & y & x\\
\hline
Motor Rickshaw, Taxi car & 1.13 & y & x\\
\hline
\textbf{Police Vehicle} & \textbf{1.50} & \textbf{y} & \textbf{x}\\
\hline
\end{tabular}
\end{table}

Of the 16000 patients in the TITCO-India data set, 1668 were adult
vehicular trauma patients who were directly admitted to the study sites.
of these complete data set was available for 1157 patients.

\hypertarget{references}{%
\section*{References}\label{references}}
\addcontentsline{toc}{section}{References}

\hypertarget{refs}{}
\leavevmode\hypertarget{ref-Purcell2020}{}%
1. Purcell LN, Mulima G, Nip E, Yohan A, Gallaher J, Charles A. Police
Transportation Following Vehicular Trauma and Risk of Mortality in a
Resource-Limited Setting. World Journal of Surgery. 2020;45:662--7.
doi:\href{https://doi.org/10.1007/s00268-020-05853-z}{10.1007/s00268-020-05853-z}.

\end{document}
